\documentclass[UTF8,a4paper,8pt]{ctexart} 

\usepackage{graphicx}%学习插入图
\usepackage{verbatim}%学习注释多行
\usepackage{booktabs}%表格
\usepackage{geometry}%图片
\usepackage{amsmath} 
\usepackage{amssymb}
\usepackage{listings}%代码
\usepackage{xcolor}  %颜色
\usepackage{enumitem}%列表格式
\CTEXsetup[format+={\flushleft}]{section}


\geometry{left=1.6cm,right=1.8cm,top=2cm,bottom=1.7cm} %设置文章宽度

\pagestyle{plain} 		  %设置页面布局
\author{郑华}
\title{C++  Abstract Class}
%代码效果定义
\definecolor{codegreen}{rgb}{0,0.6,0}
\definecolor{codegray}{rgb}{0.5,0.5,0.5}
\definecolor{codepurple}{rgb}{0.58,0,0.82}
\definecolor{backcolour}{rgb}{0.95,0.95,0.92}

\lstdefinestyle{mystyle}{
	backgroundcolor=\color{backcolour},   
	commentstyle=\color{codegreen},
	keywordstyle=\color{magenta},
	numberstyle=\tiny\color{codegray},
	stringstyle=\color{codepurple},
	basicstyle=\footnotesize,
	breakatwhitespace=false,         
	breaklines=true,                 
	captionpos=b,                    
	keepspaces=true,                 
	%numbers=left,                    
	%numbersep=5pt,                  
	showspaces=false,                
	showstringspaces=false,
	showtabs=false,                  
	tabsize=2
}
\lstset{style=mystyle, escapeinside=``}


 %正文排版开始
 \begin{document} 
 	\maketitle
 	
 	\section{概要}
	  纯虚函数是一种特殊的虚函数,它的一般格式如下:
		 \begin{lstlisting}
class <类名>
{
   virtual <类型><函数名>(<参数表>)=0;
   …
};	 	
		 \end{lstlisting}

   
   在许多情况下,在基类中不能对虚函数给出有意义有实现,而把它说明为纯虚函数,它的实现留给该基类的派生类去做。这就是纯虚函数的作用。下面给出一个纯虚函数的例子。   
		   \begin{lstlisting}
class point
{
public:
	point(int i=0, int j=0) { x0=i; y0=j; }
	virtual void set() = 0;	
	virtual void draw() = 0;
			
protected:
	int x0, y0;
};
		  
class line : public point
{
public:
	line(int i=0, int j=0, int m=0, int n=0):point(i, j)	
	{
		x1=m; y1=n;
	}	
	void set() { cout<<"line::set() called."; }
	void draw() { cout<<"line::draw() called."; }
		   	
protected:
	int x1, y1;
};

class ellipse : public point
{	
public:
	ellipse(int i=0, int j=0, int p=0, int q=0):point(i, j)
	{	
		x2=p; y2=q;	
	}
	void set() { cout<<"ellipse::set() called."; }
	void draw() { cout<<"ellipse::draw() called."; }
			
protected:
	int x2, y2;
};


void drawobj(point *p)
{	
	p->draw();
}
						  
void setobj(point *p) 
{	
	p->set(); 	
}
			   
void main()
{
	line *lineobj = new line;
	ellipse *elliobj = new ellipse;
	drawobj(lineobj);
	drawobj(elliobj);
	setobj(lineobj);
	setobj(elliobj);
	cout<<"Redraw the object...";
	drawobj(lineobj);
	drawobj(elliobj);
}
   \end{lstlisting}
   执行结果:
   
   line::draw() called.
   
   ellipse::draw() called.
   
   line::set() called.
   
   ellipse::set() called.
   
   Redraw the object...
   
   line::draw() called.
   
   ellipse::draw() called.
   
\section{抽象类}
   
   带有纯虚函数的类称为抽象类。抽象类是一种特殊的类,它是为了抽象和设计的目的而建立的,它处于继承层次结构的较上层。抽象类是不能定义对象的,在实际中为了强调一个类是抽象类,可将该类的构造函数说明为保护的访问控制权限。
   
   抽象类的主要作用是将有关的组织在一个继承层次结构中,由它来为它们提供一个公共的根,相关的子类是从这个根派生出来的。
   
   抽象类刻画了一组子类的操作接口的通用语义,这些语义也传给子类。一般而言,抽象类只描述这组子类共同的操作接口,而完整的实现留给子类。
   
   抽象类只能作为基类来使用,其纯虚函数的实现由派生类给出。如果派生类没有重新定义纯虚函数,而派生类只是继承基类的纯虚函数,则这个派生类仍然还是一个 抽象类。如果派生类中给出了基类纯虚函数的实现,则该派生类就不再是抽象类了,它是一个可以建立对象的具体类了。 
 
 		     
\end{document} 
 		    